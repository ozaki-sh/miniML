\documentclass{jarticle}
\usepackage[dvipdfmx]{graphicx}
\usepackage{here}
\usepackage{listings}

\begin{document}

\title{レポート2(インタプリタ)}
\author{1029289895 尾崎翔太}
\date{2018/6/29}

\maketitle

\section{はじめに}
ソースコード単位で解説していくのではなく、課題単位で解説していく。
また、説明はできるだけその時点における説明にしようと思うので、ソースコードとの整合性が取れない場合がある。
例えば、ソースコードにおいてLTExprはCmpExprになっているが、説明ではLTExprが登場する。最後に、課題には存在しなかったが、
デバッグのしやすさから減算と、比較のイコールを実装している(ただし、減算は-と第二オペランドの間に空白を入れないと
きちんと動作しない)。

\section{各課題の説明}
\subsection{Exercise 3.2.1}
テストプログラムとその実行は以下である。
\begin{lstlisting}[basicstyle=\ttfamily\footnotesize, frame=single]
# if (3 * 4) < 15 then 1 + 2 * 3 else 0;;
val - = 7
\end{lstlisting}
これによりif文、加算、乗算、比較、その優先順位が正しく動作していることがわかる。
また、問題文にあるようにiiを2、iiiを3、ivを4に束縛した環境を大域環境にすると
\begin{lstlisting}[basicstyle=\ttfamily\footnotesize, frame=single]
# iv + iii * ii;;
val - = 10
\end{lstlisting}
となる。

\subsection{Exercise 3.2.2}
\subsubsection{設計方針}
read-eval-printループ内でエラーを検出して、エラーの種類に応じてエラーメッセージを表示したあとに
read-eval-printループに戻る。

\subsubsection{実装の詳細}
main.mlの7〜10行目にprint\_error\_and\_go関数を定義している。これはstringを受け取って、それを表示したあとに
read\_eval\_print関数を呼び出す。これは、エラー処理をまとめたものである。そしてmain.mlの12行目にtryがあって、
47行目にwithがあって、エラーをパターンマッチして、表示するメッセージをprint\_error\_and\_go関数にわたす。
エラーの種類はevalで発生するもの、parserで発生するもの、lexerで発生するもの、正体不明の四つに分類した。

\subsection{Exercise 3.2.3}
\subsubsection{設計方針}
単純に \& \& と \textbar \textbar を認識できるようにして、それらに対応する文法規則を追加する。評価においては、
第一オペランドを評価した時点で値が確定するなら第二オペランドは評価しない。そのため、通常の二項演算子とは
異なる関数が必要となる。

\subsubsection{実装の詳細}
lexer.mllの38行目は「\& \& 」をANDというトークンに、39行目は「\textbar \textbar」をORをいうトークンに変換する
規則である。そして、優先順位を考慮して、LTExprの上にAndExpr、AndExprの上にOrExprというようになっている。評価に関しては、
他の二項演算子と違って、オペランドを評価したりしなかったりするので、区別するためにBinLogicOpという新たな型を作った。この型のコンストラクタはAndとOrの二つである。そして、eval.mlの64〜77行目にapply\_logic\_prim関数を定義している。これは、
BinLogicOpとexvalとexpを引数にとって、exvalを返す。opがAndでarg1がBoolV falseならexp2は評価せずにBoolV falseを返す。
opがAndでarg1がtrueならexp2の評価結果を返す。opがOrの場合も同様である(trueとfalseは逆だが)。なお、オペランドがboolの値をもってなければエラーを吐く。また、eval\_exp関数にBinLogicOpの場合を追加して、第一オペランドを評価してからapply\_logic\_prim関数を呼び出している。最後に、apply\_logic\_prim関数とeval\_exp関数は相互再帰の形になったので、andでつないでいる。

\subsection{Exercise 3.2.4}
\subsubsection{設計方針}
lexer.mllに新たなルールを追加して、再帰的に呼び出せるようにする。ただし、コメントのネストがきちんとできるように
コメントの深さを管理する。

\subsubsection{実装の詳細}
lexer.mllの57行目以降にcommentというルールを追加した。このルールは整数を一つ受け取る。この整数はコメントの深さを
表していて、この「($\ast$ 」が現れるたびに1増え、「$\ast$) 」が現れるたびに1減る。ルールmainで「($\ast$ 」が現れた
ときはルールcommentを0で呼び出す。そして、0のときに「$\ast$)」が現れたときはルールmainを呼び出す。

\subsection{Exercise 3.3.1}
教科書通りにしただけなので、設計方針等はない。
テストプログラムは以下である。
\begin{lstlisting}[basicstyle=\ttfamily\footnotesize, frame=single]
# let x = 1 + 2;;
val x = 3
# let y = 1;;
val y = 1
# let y = x + 1 in y * 5;;
val - = 20
# y;;
val - = 1
\end{lstlisting}
let宣言によってxが3に束縛されていることがわかり、let式もきちんと動作していることがわかる。
さらに、let式で束縛したyはスコープの外では見えないことがわかる。

\subsection{Exercise 3.3.2/Exercise 3.3.4}
\subsubsection{設計方針}
let宣言の列はandでつながれたlet宣言の列の列である、と考える。よって、let宣言の全体像としては、リストのリストのようになる。
また、andの方は評価に使う環境と束縛を付加する環境が違うことに注意する。let式の方は、let宣言と大差ないので、let宣言を実装してから
それをいじって実装することにする。
\subsubsection{実装の詳細}
主にlet宣言について説明する。syntax.mlの45行目のように、let宣言を表すコンストラクタはDecls of ((id $\ast$ exp) list) listである。
設計方針で述べたように、外側のリストがlet宣言の列、内側のリストがandでつながれたlet宣言を表している。
lexerについては「and」をLETANDというトークンに変換する規則を追加した。
parserはparser.mlyの28〜36行目にLetDeclとLetAndDeclという非終端記号を追加している。
中身はどちらも同じような格好をしていて、打ち切るか続くかのどちらかとなっている。
評価についてはeval.mlの214〜234行目で行っている。複雑だが、要するに((id $\ast$ env $\ast$ value) list) listを返している。
外側のループは単に外側のリストを構成しているだけで、いろいろしているのは内側のループである。内側のループは、andでつながれた
let宣言において同じ変数に束縛していないかを、変数名のリストを持ちまわることでチェックしている。また、式の評価は外側のループに
おける環境で行っていて、その束縛を付加する環境は内側のループのものである。そして、内側のループを抜けるときにその環境を外側の
ループの環境に代入する。要するに、andでつながれたlet宣言によって起こった束縛はまとめて環境に付加するということである。
main.mlにおいては15〜46行目が関係している部分である。24行目まではconcatしているだけである。ただし、それらをすべて付加された
環境も一緒に返している。その後、それを表示していくのだが、let宣言の列で同じ変数に複数回束縛している場合は、一番最後の束縛の
結果のみを表示する必要があるので、変数名を見ながら、要らないものは削除している。let式については、eval.mlの117〜129行目にあるが、
andでつながれたものしかないので、ループが一つで済んですっきりしている。

\subsection{Exercise 3.3.3}
\subsubsection{設計方針}
シンプルにやろうとして上手くいかなかったので、ファイルの中身を全部stringにして、;;で区切ってリストにして順番に
lexerに与えていくことにする。また、ファイル中のプログラムにの実行中にエラーがあった時は、そのプログラムは実行は
中断するが、その後に他のプログラムがあった場合、それらは実行される。
\subsubsection{実装の詳細}
まず、main.mlの139行目以降で、ファイル名が与えられているならファイル用のread-eval-printループへ、
そうでないなら通常のread-eval-printループへ飛ぶようにしている。
本体は54〜128行目であるが、81行目以降は通常のものと大差ない。関数を上から見ていくと、まず、ファイルを開いて、
空文字列への参照strを作っている。そして、ファイルの中身を一行ごとに空白で区切りながらstrの中身と結合していく。
ファイルを読み終えるとファイルを閉じて、strの中身を;;で分割してリストにしていく、これはget\_str\_list\_by\_semisemi関数で
行っており、一見複雑だが、要するに左から見ていって;;にであったところで切り出しているだけである。なお、;;も一緒に切り出さないと、
parserがプログラムと認識してくれないのでそうしている。そうしてできたリストの中身を順にlexerに渡していくことでファイルの中身を実行
している。リストが空になると、ファイルの実行が終わったことを表示して、通常のread-eval-printループへ移行する。

\subsection{Exercise 3.4.1}
教科書通りにしただけなので、設計方針等はない。教科書に載っていないparser.mlyのFunExprの部分も特筆することはないほど単純である。
pp\_valについても、$<$fun$>$と表示するようにしただけである。
テストプログラムは以下である。
\begin{lstlisting}[basicstyle=\ttfamily\footnotesize, frame=single]
# let y = 10;;
val y = 10
# let apply = fun f -> fun x -> f (x + y);;
val apply = <fun>
# let y = 5;;
val y = 5
# apply (fun x -> 2 * x) 10;;
val - = 40
\end{lstlisting}
これにより、高階関数がきちんと動作していること、関数定義時にきちんと関数閉包が作られていることがわかる。

\subsection{Exercise 3.4.2}
\subsubsection{設計方針}
(演算子)を認識する新たな文法をparserに追加する。当然これを関数適用に使いたいので優先度はAppExprとAExprの間とする。
アクションは単にFunExpを返すだけでよい。
\subsubsection{実装の詳細}
parser.mlyの104〜112行目にFunInfixExprという新たな非終端記号を追加した。それぞれ、仮引数の名前を適当に決めて、
その演算を行うだけの関数を返す。

\subsection{Exercise 3.4.3}
\subsubsection{設計方針}
複数引数の関数は、カリー化の考えで、1引数関数に帰着できるので、parser.mlyだけをいじればよい。
上の方は単にparserにおいて引数の部分をリストで受け取るようにして、アクション部でそれを展開すればよい。
下の方はあくまでletによる表現だとしてLetAndExprを拡張するような形で実装する。
\subsubsection{実装の詳細}
上の方はparser.mlyの150〜156行目の部分である。パラメータをnonempty\_list関数を通してもらうことで、1つ以上受け取ることができる。
アクション部も、単にFunExpの中にFunExpができるようにしてカリー化を実現しているだけである。
下の方はparser.mlyの142〜148行目にLetFunExprというものを追加した。中身は前者とほぼ同じである。これをLetAndExprでExprの代わりに
受け取れるようにしておけばよい。

\subsection{Exercise 3.4.5}
\subsubsection{設計方針}
「dfun」を予約語にして、parserの部分では通常の関数と同じように処理すればよい。
evalではProcVから環境を取り除いたDProcVを追加して、AppExpの評価時に関数がDProcVの場合は今の環境にパラメータの束縛を与えて
本体を評価すればよい。
\subsubsection{実装の詳細}
設計の方針をそのまま単純に実装しただけで、あまり説明することはない。一つ説明するなら、DProcVのpp\_valの結果は$<$dfun$>$にした。

\subsection{Exercise 3.5.1}
教科書通りにしただけなので、設計方針等はない。実装するときに教科書そのままでは何か問題が起こった気がするが、覚えていない。
テストプログラムは以下である。
\begin{lstlisting}[basicstyle=\ttfamily\footnotesize, frame=single]
# let rec fact n =
    if n < 1 then 1
    else n * (fact (n - 1));;
val fact = <fun>
# fact 5;;
val - = 120
# let rec sum n =
    if n < 1 then 0
    else n + (sum (n - 1))
  in
    sum 10;;
val - = 55
\end{lstlisting}
これにより、let rec宣言、式ともに正しく動作していることがわかる。

\subsection{Exercise 3.5.2}
\subsubsection{設計方針}
Exercise 3.3.2と3.3.4の場合と同じように実装した。ただし、andに関しては振る舞いが異なる。andでつながれたlet recで束縛するProcVの中の
環境への参照が同じ環境を指している必要があるので、単純にループで同じ環境を共有していくことにした。
\subsubsection{実装の詳細}
parserはExercise 3.3.2と3.3.4の場合とほぼ同じなので割愛する。evalについても評価する環境と束縛を与える環境が同じになったので、
むしろ単純である。同じ環境を共有するとか考えている内に、どうせ上書きされるのでわざわざdummyenvを作らなくてもよいことに気づいたので、
envをそのまま使いまわしている。
\subsubsection{テストプログラム}
テストプログラムは以下である。これはObjective Caml入門に載っていた相互再帰関数の例である。
\begin{lstlisting}[basicstyle=\ttfamily\footnotesize, frame=single]
# let rec even n =
    if n = 0 then true else odd(n - 1)
  and odd n =
    if n = 0 then false else even(n - 1);;
val even = <fun>
val odd = <fun>
# even 6;;
val - = true
# odd 14;;
val - = false
\end{lstlisting}
きちんと動作していることがわかる。

\subsection{Exercise 3.6.1}
match式については2.16節で述べる。
\subsubsection{設計方針}
まず、expにリストを表すコンストラクタを作る。そして、新たな型としてlistExpというものを定義する。listExpは二つのコンストラクタを持っていて、
一つは空リストを表し、もう一つは先頭の要素と残りのリストの組を表す。これをexpのリストのコンストラクタに持たせておく。同様のことを
exvalについても行っておく。lexerとparserの部分は単純に実装すればよい。evalについては評価そのものは簡単である(expとexvalが同じような
形を持つようにしたので)。::の演算も単純に実装すればよくて、一番複雑なのはpp\_valするためにstringに変換する部分である。これは
中身の要素を順に見ていく形で実装した。
\subsubsection{実装の詳細}
syntax.mlの32行目と41行目が関係している。設計方針で述べた通りに実装している。lexerについては、「[]」で一つのトークンを作ろうと
思ったが、次の課題を踏まえて左かっこと右かっこそれぞれがトークンを作るようにした。parserについては、空リストはAExprの一部として、
ConsExprはMExprより弱く、CmpExprよりは強くした。evalはeval.mlの13、14行目にリストに対応する値を追加した。23〜38行目の
string\_of\_list関数でListVをstringに変換している。この関数は、まず空リストなら「[]」に変換する。そうでなければ、「; arg1; arg2; ...]」の
形を作って、最後に頭二文字を削って「[」を結合する。評価の部分は157〜165行目だが、単純に要素を順番に評価しながら対応するListVを構築していく。

\subsection{Exercise 3.6.2}
\subsubsection{設計方針}
parserをいじるだけでよい。AExprに追加するような形で実装する。始まりの「[hoge;」があって「fuga;」で続いていくか「]」で終わるかのどちらか
であると考えて実装する。
\subsubsection{実装の詳細}
parser.mlyの123〜128行目が関係している。設計方針で述べた通りである。ここで返すのはConsであって、ListExpはAExprが返してくれる。

\subsection{Exercise 3.6.3/Exercise 3.6.4}
\subsubsection{設計方針}
少しずつ拡張していくようにしたがこれは失敗で、新しい機能を追加するたびに元のプログラムのほとんどを変更していた。完成したものに対する設計方針
を述べる。まず、parserにパターンに関する文法を追加する。そして、evalにパターンマッチする関数を作っておいて、match式の評価に利用する。
match式に対応するexpのコンストラクタはマッチされる式のリストと、パターンのリストと本体式の組のリストを持つ。evalでは、マッチされる式をまず
順番に評価して、値のリストを得た後に、パターンマッチをして、マッチすれば本体式を評価する。また、各パターンで同一の変数が使われているか
をチェックする。さらに、ワイルドカードも実装する。
\subsubsection{実装の詳細}
考え方はsyntaxc.mlの35〜39行目のコメントを見ていただけるとわかると思う。parserはparser.mlyの177行目以降が関係している。一つ以上を許すものが
多いので、More...という導出規則が多くて見づらくなっているが、そこを無視すればわかりやすくなる。APatternというのはAExprのようなもので、
最も単純なパターンである。Patternというのはリストに関するパターンを記述できるようにしたものである。PatternMatchExprというのは、
(パターン) -$>$ (本体式) という部分を表している。MatchExprはmatch文全体を表している。次に、eval.mlの50〜62行目のpattern\_match関数について
説明する。これは、パターンのリストとマッチされる式のリストを受け取って、マッチしたなら束縛すべき変数と値の組のリスト、マッチしなかったなら
MatchErrorというエラーを投げる関数である。マッチしているかどうかの判断は単純に二つのリストを頭から順に見ていっているだけである。
ただし、パターンのリストの型はexp listで、マッチされる式のリストの型はexval listであることには気をつける必要がある。実際に評価する部分は
eval.mlの166〜209行目で、少しコメントを付けてあるので、ここで説明するよりもそちらを見ていただいた方がわかりやすいと思う。
andでたくさん関数が宣言してあるが、他の関数を呼び出しているのはouter\_loop関数だけなので、実はあまりandでつなぐ意味はなく、
トップダウンに考えて実装したことが表れているだけである。

\subsection{Exercise 3.6.5}
\subsubsection{設計方針}
各二項演算の右側に、「できるだけ右に延ばして読む」構文が来ても良いようにを拡張する。そのために、「できるだけ右に延ばして読む」構文を
一つにまとめる。そして、それを各二項演算の右側に追加する。
\subsubsection{実装の詳細}
parser.mlyの58〜64行目に「できるだけ右に延ばして読む」構文をまとめたLookRightExprを追加した。そして、各二項演算の右側に
それを追加しただけである。

\section{感想}
難しい課題と簡単な課題の差が大きいと思った。特にパターンマッチの実装はすごく難しかった。その結果、実装がちょっと複雑になったので、
このレポートで説明することを放棄してしまった。ただ、Exercise 3.6.4は、一般的なパターンマッチという言葉が抽象的、かつ、
Exercise 3.6.1で実装したパターンマッチとocamlのパターンマッチの差が大きい、ということで何を実装すればよいのかがわかりづらいのは
どうなのかと思った。実際、ワイルドカードを実装した後と前で通ったテストの数が変わらなかったりした(他の機能がなかったせいかもしれないが)。
ともあれ、インタプリタの作成は、プロセッサの作成よりもずっと面白いと感じた。




\end{document}
